\chapter*{Prehistory}

\section*{1660}

In the 1660s~\cite[p3]{leary}, in the reign of Charles II, there were records of two troops of volunteer Light Horse in the county of Cheshire. One record from 1660~\cite[p289]{leary} describes a troop from the Hundreds\footnote{A \emph{hundred} is a subdivision of a county dating from before the Norman conquest - Cheshire had between seven and twelve hundreds at different times} of Wirral, Bucklow, and Macclsesfield, led by George Warburton, and a record from 1666~\cite[p285]{leary} describes a troop from the hundreds of Broxton, Northwich, Nantwich, and Eddisbury\footnote{These are the modern names - contemporary records use \textit{Namptwich} and \textit{Edesbury}}, led by Lt~Col~Sir~Philip Egerton.

It's not clear how these troops connected at all to the later raising of the Cheshire Yeomanry proper, but the names associated with the two troops are familiar through the whole history of the Cheshire Yeomanry, including Cholmondeley, Egerton, Grosvenor, Wilbraham, and Warburton. It's notable that the two troops were led by an Egerton and a Warburton, and the last serving Ergerton-Warburton was in the Regiment as late as the 1960s.
